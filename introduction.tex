%!TEX root=dissertation.tex
\chapter*{Введение}							% Заголовок
\addcontentsline{toc}{chapter}{Введение}	% Добавляем его в оглавление

Современные волоконные лазеры являются неоспоримыми лидерами по стоимости, технологичности, качеству излучения среди других типов лазеров. Задача создания мощных волоконных лазеров киловаттного и мультикиловаттного уровней является крайне важной для обеспечения как технологической, так и военной безопасности страны. На пути ее решения встает еще более принципиальная задача обеспечения производства отечественной элементной базой начиная с лазерных чипов и полупроводниковых модулей накачки, кончая оптическим волокном и волоконными элементами лазера. Однако, без понимания физики происходящих в них процессов, а также освоения экспериментальных методов их создания, наличия современной технологической базы получить качественную и эффективную элементную базы мощных волоконных лазеров представляется невозможным.

Волоконные лазеры и усилители проделали длинный путь с тех пор как Элиасом Сницером в начале 1960-ых был продемонстрирован первый волоконный лазер~\cite{into_las_1}. В первые годы росту волоконных лазерных технологий препятствовало отсутствие соответствующих источников накачки. Появление одномодовых диодов накачки ознаменовало начало эпохи <<ренессанса>> в исследовании волоконных лазеров. В 1986 исследователи продемонстрировать первые практические волоконные лазеры, легированные ионами эрбия~\cite{into_las_2}.

Однако это были одномодовые устройства, которые в свою очередь требовали, чтобы источники накачки также были одномодовыми, таким образом значительно увеличивалась стоимость системы и ее сложность. В 1988 была представлена лазерная система с накачкой в оболочку~\cite{into_las_3} посредством чего излучение накачки может быть ведено в значительно большую по диаметру внутреннюю оболочку волокна. Это значительно снизило нагрузку при накачке мощных волоконных лазеров.  Прорывные разработки в технологии оптического волокна позволили увеличить мощность непрерывных волоконных лазеров почти на 2 порядка за десять лет и достигнуть киловаттного уровня. Это в свою очередь повышает потенциал волоконных лазеров как предпочтительного лазерного  источника на замену большинству современных лазеров, а также значительно расширяет область их применений.

Использование волоконных лазерных устройств широко и разнообразно. Их приложения включают сваривание, абляцию, резку, отжиг, спекание, закалку и маркировку. Волоконные лазерные источники --- универсальные, эффективные, и все более и более важные инструменты во многих отраслях промышленности от оборонного производства до медицины, от пищевой промышленности и упаковки до электроники и обработки полупроводников.

Наблюдается всевозрастающий рост спроса на коммерческие волоконные лазеры большой мощности и высокой яркости. Фактически, глобальный лазерный рынок, в настоящее время оцениваемый в \$1,5 миллиарда, испытывает рост в 20~\% ежегодно начиная с 2010 года~\cite{into_las_16}.  Для того, чтобы не отставать от всевозрастающих требований оборонной и обрабатывающей промышленности, стимулировать расширение существующих и поиск новых приложений, а также закрепить позиции волоконных лазеров среди других типов лазерных источников крайне важно дальнейшее масштабирование выходной мощности волоконных лазеров с сохранением высокого качества пучка.

Волоконные лазерные системы обладают рядом преимуществ: они достаточно эффективны, как электрооптически (эффективность <<от розетки>>), так и оптическо-оптически (квантовая эффективность). Так как волноводность и усиление обеспечивается активной легированной сердцевиной волокна, качество выходного пучка волоконных лазеров обычно очень высокое. Одномодовые волокна в этом отношении особенно выделяется, обеспечивая не только распространение единственной поперечной моды, но также они значительно менее чувствительны к наведенным тепловым модовым искажениям, чем обычные твердотельные лазерные системы.
% TODO: найти ссылку в подтверждение слов по тепловым модовым искажениям

Гибкость, устойчивость к растяжению и защищенность оптоволокна означают, что составные волоконные элементы лазерной системы могут быть смотаны до относительно малого диаметра и компактно упакованы. Кроме того, усовершенствования методов соединения (сварки) оптических волокон и волоконных элементов позволило создавать монолитные всеволоконные лазерные системы с накачкой и передачей излучения генерации только по волокну. Это увеличило мобильность и надежность к внешним воздействиям, таким как вибрация, удары или значительные колебания температуры. Все это делает волоконные лазерные источники относительно простыми в производстве и легкими в обслуживании, что крайне привлекательно для интеграции их в промышленные обрабатывающие устройства, где неприменимы хрупкие, малоэффективные и большими системы.

В отличие от объемных твердотельных лазеров, которые обычно используют кристаллические активные элементы, волоконные лазеры и усилители имеют кварцевую основу, что означает наличие широкого эмиссионного спектра и, следовательно, большого диапазона рабочих длин волн. Это, а также наличие высокого коэффициента усиления в активном волокне, означают, что эти лазерные источники могут работать в широком спектральном диапазоне с высокой эффективностью. Множество возможных комбинаций основы волокна и легирующих элементов позволяет еще больше расширить спектральный диапазон работы волоконных лазеров вплоть до 2~мкм.

Высокое значение соотношения поверхность-активный объем приводит к высокой степени диссипации тепла с поверхности волокна, а его большая длина означает, что тепловая нагрузок распределена на большом расстоянии. Кроме того, использование спектральных возможностей активного волокна, а именно накачки в зоне близкой к генерации снижает квантовый дефект и вызванную им тепловую нагрузку. Все это приводит к низким требованиям к системе охлаждения лазера. Малые размеры сердцевины волокна и большие длины позволяют добиться очень высоких значений усиления, делая их идеальными для создания мощных волоконных лазеров. Типичные волоконные лазеры большой мощности устроены в конфигурации, известной как MOPA (задающий генератор -- усилитель мощности), в котором выходное излучение лазера малой мощности и высокого качества (ЗГ) усиливается, проходя через несколько каскадов усиления и достигает заданной величины мощности, которая в случае непрерывного испускания составляет порядка нескольких сотен ватт. Далее эти модули могут быть соединены вместе, чтобы получить еще более высокую суммарную мощность в несколько киловатт.

Непосредственно ЗГ, как правило, представляет собой волоконный лазер (хотя возможны и другие варианты, например, одномодовый полупроводниковый лазер с волоконным выходом). При этом оптоволоконный усилитель мощности (ОВУМ) представляет собой каскад волоконных усилителей, сделанных из оптического кварцевых волокон, легированных редкоземельными элементами, и специальной структуры, такой как большая площадь моды (LMA) с двойной оболочкой или фотонно-кристаллическая внутренняя структура.

Критическим узлом как для ЗГ, так и для ОВУМ является оптоволоконный объединитель накачки (ОВОН). Последний должен гарантировать эффективный ввод излучения, испускаемого полупроводниковыми лазерными диодами высокой яркости в активное волокно, избегая оптических повреждений и обратного отражения излучения. Объединители могут быть двух типов:
\begin{itemize}
  \item Стандартный ОВОН (также известный как ОВОН без сигнального канала): все входные порты сделаны из волокна одного тип и используются для ввода излучения накачки в активное волокно; в случае активного волокна с двойной оболочкой --- наиболее распространенный случай --- накачка вводится в первую оболочку. Обозначается как N:1, где N -- количество объединяемых каналов.
  \item ОВОН с сигнальным каналом (feedthrough): центральное волокно в жгуте --- пассивное волокно с сердцевиной, равной по размеру сердцевине активного волокна, формирующей непрерывный канал распространения сигнала. В идеале, любой обратный сигнал попадает в сердцевину пассивного волокне, а не в волокна накачки, таким образом обеспечивая автоматическую защиту диодов накачки. ОВОН с сигнальным каналом необходимы для накачки каскадов, формирующих ОВУМ. Они также часто используются в ЗГ для предохранения диодов накачки от оптического сигнала, проходящего через зеркала резонатора, так как даже <<глухое>> зеркало пропускает 1-5~\% излучения генерации. Обозначается как N+1:1, где N -- количество объединяемых каналов.
\end{itemize}

Внутреннее строение волоконных объединителей накачки имеет структуру тейпера, которая изготавливается путем размещения заданного числа смежных каналов рядом друг с другом (встык по боковой стороне), и затем их сплавления и растяжения, чтобы создать центральную область связи. На практике процедура более комплексна не только потому, что волокна должны быть удержаны в правильном положении, но также и потому что необходимы предварительные операции, такие как предварительное сужение или травление в плавиковой кислоте. ОВОН с сигнальным каналом особенно трудно изготовить, потому что волокно сигнального канала должен быть сохранен в центре жгута с высокой точностью на каждом шаге изготовления. Кроме того, геометрические ограничения (плотная упаковка) позволяют изготавливать ОВОН, основанных на структуре тейпера, только с определенным числом входных каналов (наиболее распространены 7, 19, и 91), тогда как закон сохранение яркости налагает ограничения на числовую апертуру входного волокна. Большое число ОВОН доступны коммерчески как стандартные волоконные элементы, хотя и с небольшим количеством входных каналов, и для небольшого числа вариаций типов и геометрии волокон. Однако, во многих случаях, и в особенности для ОВОН с сигнальным каналом, готовые продукты часто не соответствуют необходимым требованиям. Для устойчивой и корректной работы волоконного лазера ОВОН должны быть качественно состыкованы с активным волокном, согласованы по числовой апертуре, модовому распределению и геометрии.
аким образом, потребность в овладении методами создания волоконных объединителей накачки с хорошей повторяемостью крайне высока. В представленной работе приведены результаты разработки такой методики; представлены усовершенствования существующих техник изготовления и оптические характеристики произведенных волоконных элементов. Производительность полученых волоконных объединителей накачки протестирована в изготовленном на из основе волоконном лазере киловаттного уровня выходной мощности.

Еще одним неотъемлемым элементом волоконных лазеров является волоконные брэгговские решетки (ВБР). В первую очередь они используются в волоконных лазерных системах в качестве зеркал волоконного резонатора, а также в качестве чувствительных оптоволоконных элементов в агрессивной окружающей среде, где часто они должны  сохранять свои физические свойства при очень высоких температурах~\cite{into_fbg_1,into_fbg_2,into_fbg_3}.

Хотя ВБР обычно воспринимается как дифракционная решетка с <<постоянной>> (неизменной) модуляцией показателя преломления в сердцевине волокна, их эксплуатация в режимах, подразумевающих высокие температуры, может привести к деградации (стиранию) наведенной модуляции показателя преломления, формирующей решетку. Максимальная температура, которую может поддерживать условная ВБР (используемая, например, в качестве температурных датчиков или при высоких энергиях проходящего оптического излучения), как правило составляет не более 600~$^\circ$C, что обусловлено слабостью химической связи германия и кислорода внутри волокна~\cite{into_fbg_4}. Был проведен ряд научно-исследовательских работ с целью увеличить термическую стойкость структур решетки, например, посредством экспериментов, вовлекающих ускоренное старение, предварительное облучение, формирование решеток типа II, легирование волокна специфическими химическим элементами и т.д.~\cite{into_fbg_5,into_fbg_6,into_fbg_7,into_fbg_8,into_fbg_9}. Однако, высокий уровень деградации коэффициента отражения решетки, который недопустим во многих прикладных задачах все еще имеет место при высоких температурах. Недавно, сообщалось о восстановленных решетках, произведенных на волокне, легированном B/Ge, которое может поддерживать высокие температуры свыше 1200~$^\circ$C~\cite{into_fbg_10,into_fbg_11}. Однако, для этого типа решеток требуется волокно, легированное специфическими элементами, а также наводораживание (насыщение оптического волокна водородом под давлением свыше 150~атм.)~\cite{into_fbg_10}.

За последние несколько лет значительное внимание было уделено отработке процесса изготовлению ВБР при помощи импульсов фемтосекундных лазеров. Полученные критерии термической устойчивости показывают, что записанные таким образом решетки типовые II-IR показывает превосходную стабильность при температурах свыше 1000~$^\circ$C~\cite{into_fbg_12,into_fbg_13}. Это объясняется результатом нелинейного процесса самофокусировки, где крайне высокая пиковая мощность лазерного излучения локально воздействует на структуру матрицы стекла~\cite{into_fbg_2,into_fbg_14}. Однако, хорошая термическая устойчивость не сохраняется при росте температуры свыше 1100~$^\circ$C~\cite{into_fbg_12,into_fbg_13}. В этом случае решетки быстро разрушаются. Одно из ограничений в достижении лучшей термической устойчивости заключается в остаточном напряжении, существующем в оптоволокне во время процесса формирования решетки, которое отрицательно сказывается на надежности волокна, качестве решетки и термической устойчивости. Остаточное напряжение возникает, главным образом, из-за суперпозиции теплового напряжения, вызванного разностью коэффициентов теплового расширения между сердцевиной волокна и его покрытием, а также механическим напряжением, вызванным разностью в вязкости и упругости этих двух областей~\cite{into_fbg_15}.

В представленной работе приведены результаты исследований термической устойчивости изготавливаемых оптических волокон с специфическими легирующим элементами, осуществлен подбор термостойкого оптического покрытия, а также сформулирован и отработан процесс записи ВБР на волокне с большим полем моды и двойной оболочкой. Полученные решетки показывают превосходную термическую устойчивость при температурах до 1200~$^\circ$C, показывая устойчивость коэффициент отражения решетки и резонансной длины волны в течение 20 часов изотермических измерений. Кроме того, у решеток, представленных в данной работе, есть высокий потенциал их применения в одноволоконных (однокаскадных) лазерных системах с выходной оптической мощностью свыше 1 кВт.

\textbf{Целью} данной работы является разработка и реализация процесса создания эффективных интерференционных и волоконных элементов из собственного оптического волокна, а также их апробация при создании волоконного лазера с выходной мощностью до 1 кВт на отечественной элементной базе.

Для~достижения поставленной цели необходимо решить следующие задачи:
\begin{enumerate}
  \item Проанализировать достоинства и недостатки существующих современных методов создания волоконных лазеров киловаттного уровня выходной мощности на предмет используемой волоконно-оптической элементной базы. Выбрать наиболее подходящий метод для его последующей реализации в рамках решаемых задач отделения.
  \item Исследовать изменение оптических характеристик, выявить критические параметры, влияющие на результат процесса создания сплавленных волоконных элементов (объединителей накачки).
  \item Исследовать необходимость достижения высоких значений наведенного показателя преломления для повышения устойчивости ВБР при больших мощностях проходящего излучения. Проанализировать решения уравнений распространения электромагнитной волны для произвольной периодической функции в случае, если периодические неоднородности среды нельзя считать малыми.
  \item Исследовать нагрев и деградацию волоконных брэгговских решеток при больших мощностях проходящего излучения, смещение спектра, влияние большого диаметра волокна и наличие второй оболочки на процесс записи ВБР
  %\item Сконструировать и реализовать экспериментальные установки (стенды) для создания СВЭ и формирования периодических структур в оптическом волокне.
  \item Апробировать характеристики полученных оптоволоконных элементов в конструкции оптоволоконного лазера мощностью до 1 кВт.
\end{enumerate}

\textbf{Основные положения, выносимые на~защиту:}
\begin{enumerate}
  \item Конструкция компактной экспериментальные установки (стенда) для создания сплавленных волоконных элементов (объединителей накачки).
  \item Конструкция компактной экспериментальные установки (стенда) для формирования периодической структуры наведенного показателя преломления с высокой термической устойчивостью в оптическом волокне.
  \item Конструкция волоконного лазера с выходной мощностью 1 кВт на полностью отечественой элементной базе
  \item Научно-технические основы ....
\end{enumerate}

\textbf{Научная новизна:}
\begin{enumerate}
  \item Впервые в России созданы оптоволоконные объединители накачки для волоконных лазеров мощностью 1 кВт.
  \item Впервые в России выполнен анализ и оптимизация технологических параметров записи волоконных брэгговских решеток для волоконных лазеров мощностью 1 кВт.
  \item Впервые в России создан волоконный лазер на полностью отечественной элементной базе мощностью 1 кВт.
\end{enumerate}

\textbf{Научная и практическая значимость}

В 2014 году компания IPG Photonics (США) создала волоконный лазер с выходной мощностью 100 кВт. Частная военная компания Lockhead Martin (США) в 2013 создала волоконный лазер с некогерентным сложением излучения мощностью 10 кВт для военного применения. На данный момент в России с использованием зарубежной элементной базы созданы волоконные лазеры с выходной мощностью в несколько сотен ватт. Дальнейшее увеличение мощности отечественных лазеров в первую очередь ограничено отсутствием или сложностью и дороговизной приобретения волоконных элементов лазера и элементов накачки, а также отсутствием собственной технологии производства большинства волоконных элементов. Таким образом, стратегическую важность решения задачи получения собственной элементной базы мощных волоконных лазеров трудно переоценить.

\begin{enumerate}
  \item Освоение отечественного (собственного) производства объединителей накачки высокой мощности позволит сделать качественный шаг вперед в создании волоконных лазеров с выходной мощностью свыше 1 кВт для военного применения.
  \item Экспериментальное исследование таких явлений как нагрев и деградация решеток при больших мощностях проходящего излучения, смещение спектра, влияние волокон большого диаметра на процесс записи ВБР позволит в будущем однозначно оценивать возможное изменение характеристик создаваемых мощных волоконных лазеров в различных условиях эксплуатации.
  %\item Создание универсальной стандартизованной и устойчивой к внешним воздействиям конструкции упаковки различного типа волоконных элементов лазера позволит продвинутся в направлении создания стандарта конструкции волоконных лазеров с военной приемкой.
\end{enumerate}

\textbf{Степень достоверности}

Результаты находятся в соответствии с результатами, полученными другими авторами. Характеристики изготовленных элементов подтверждены при их интеграции в структуру созданного волоконного лазера.

\textbf{Апробация работы.}

Основные результаты работы докладывались~на:

Конференциях НИО-5:
\begin{enumerate}
  \item Т. Д. Чудинова, Д. В. Чудинов, Влияние фоточувствительности материала волокна на выходные характеристики волоконных брэгговских решеток, 2014
  \item Д. В. Чудинов, А. В. Маракулин, Л. А. Минашина, Технология получения объемных брэгговских решеток, 2014
  \item Д. В. Чудинов, А. М. Поляков, Разработка волоконного лазера с распределенной накачкой киловаттного уровня, 2014
  \item Т. Д. Чудинова, А. М. Поляков, Д. В. Чудинов, Отработка технологии изготовления фотоиндуцированных решеток показателя преломления методом фазовой маски, 2015
\end{enumerate}


\textbf{Личный вклад.} Все результаты исследований, описанных в диссертации, получены лично автором или под его руководством.

% БЛАГОДАРНОСТЬ: сотрудникам лабораторий 57-2 и 57-5, а также конструкторам отдела 56 НИО-5.

\textbf{Публикации.}

\begin{enumerate}
  \item Т. Д. Чудинова, Д. В. Чудинов, Методы изготовления фотоиндуцированных решеток показателя преломления, Отчет о НИР ПС 10.11789.
  \item Д. В. Чудинов, Влияние фоточувствительности материала волокна на выходные характеристики волоконных брэгговских решеток, Отчет о НИР 2011.
  \item А. М. Поляков, Д. В. Чудинов, Расчет спектра отражения брэгговской решетки. Отчет о НИР ПС13.13133.
  \item А. М. Поляков, Т. Д. Чудинова, Д. В. Чудинов, Экспериментальная отработка изготовления фоточувствительных решеток показателя преломления методом фазовой маски, Отчет о НИР ПС 13.13134.
  \item Д. В. Чудинов, А. М. Поляков, Расчет волоконного лазера с распределенной накачкой киловаттного уровня. Отчет о НИР ПС 14.13381.
  \item Запись волоконных брэгговских решёток на фоточувствительном волокне, изготовленном в лаборатории 57-5 НИО-5 РФЯЦ-ВНИИТФ, Протокол 057-04/6220 от 30.09.15.
  \item Д. В. Чудинов, Е. С. Ивченко, Т. Д. Чудинова, Ю. В. Ивченко, Отработка технологии изготовления волоконных объединителей в интересах создания мощных ОВЛДН. Протокол.
\end{enumerate}

Основные результаты по теме диссертации изложены в ХХ печатных изданиях~\cite{},
Х из которых изданы в журналах, рекомендованных ВАК~\cite{},
ХХ --- в тезисах докладов~\cite{}.

\textbf{Объем и структура работы.} Диссертация состоит из~введения, четырех глав, заключения и~двух приложений. Полный объем диссертации составляет 159~страница с~72~рисунками и~11~таблицами. Список литературы содержит 185~наименований.

\clearpage
